\documentclass{article}
\usepackage{hyperref}
\usepackage[frenchb]{babel}

%https://perso.imt-mines-albi.fr/~gaborit/latex/latex-in-french.html
%\usepackage[english,francais]{babel}
%\usepackage[english,french]{babel}

\usepackage[frenchb]{babel}

%https://tex.stackexchange.com/questions/248788/the-very-basics-of-french-accents
%\documentclass[12pt]{amsart}

\usepackage[utf8]{inputenc}

%entiers sets https://texblog.org/2007/08/27/number-sets-prime-natural-integer-rational-real-and-complex-in-latex/
%\usepackage{amsfonts} 
% or 
\usepackage{amssymb}

%Majuscules accentuees https://www.xm1math.net/doculatex/caracteres_speciaux.html
\usepackage{eurosym}
\usepackage{systeme}
\usepackage{enumerate}
%numerotation https://www.xm1math.net/doculatex/structure.html
\renewcommand{\thesubsection}{\Roman{subsection}}

\begin{document}
Hello world \LaTeX

Sommaire + index
%https://perso.imt-mines-albi.fr/~gaborit/latex/latex-in-french.html
%\renewcommand{\contentsname}{Sommaire}


\part{Équations diophantiennes du
%
1\ier{}  degré $a \cdot x+b \cdot y=c$. Autres exemples d'équations diophantiennes.}

\textbf{Déf 1}
On appelle équation diophantienne à \textit{n} inconnues, une équation du type $P(Y_{1},....Y_{n})=0$ avec
%https://tex.stackexchange.com/questions/261693/latex-element-of-with-two-strokes-%E2%8B%B9

$P \in \mathbb{Z}[X_{1}...X_{n}]$. On cherche les solutions dans
%


$\mathbb{Z}^{n}$.
\subsection{Équations diophantiennes linéaires}
%systeme d equations https://kogler.wordpress.com/2008/03/21/latex-multiline-equations-systems-and-matrices/
%https://en.wikibooks.org/wiki/LaTeX/Advanced_Mathematics
% $a_{11}  \cdot x_{i} + ... + z a_{1m}  \cdot x_{m} = b_{1}$
\[\begin{cases} a = 2 \\  c = 3 \\ d = 5 \end{cases}\]


%Ia
\subsubsection{Équations diophantiennes du $1^{er}$ degré à 2 inconnues $a \cdot x+b \cdot y=c$ (*1) .}
fgfgfggf



\textbf{Prop 1}
On appelle équation diophantienne à \textit{n} inconnues, une équation du type $P(Y_{1},....Y_{n})=0$Une condition nécessaire et suffisanted'existence d'au moins 1 solution de (*1) est pgcd(a,b) divise c.


\textbf{Théorème de Bezout}
a,b sont 2 entiers. a et b sont premiers entre eux ssi il existe (u,v) $\in \mathbb{Z}$

\textbf{Prop 2}

\textbf{Méthodes de résolution}


%Ib
\subsubsection{Systèmes d'équations diophantiennes linéaires}
Soit (m,n) $\in \mathbb{Z}$

\begin{enumerate}
        \item Résoudre le système (S) :  \systeme{7x-6y=12,5x+3y=11}
       \item Résoudre le système (S) :  \systeme{7x-6y=12,5x+3y=11}
\end{enumerate}


%II
\subsection{Équations diophantiennes et décomposition en facteurs premiers}

%III
\subsection{Équations diophantiennes et corps de nombres quadratiques}
\textbf{Équation de Fermat pour n=3}


%IV
\subsection{Équations diophantiennes et fractions continues}





\url{https://linuxconfig.org}

%https://www.ljll.math.upmc.fr/hecht/ftp/old/InfoBase-2005-06/latex/DOC/LaTex-initiation.pdf (notes de bas de pages + marges)
\footnote{Written by Peter MOUEZA}
\end{document}
